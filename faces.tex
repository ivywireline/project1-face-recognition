%%%%%%%%%%%%%%%%%%%%%%%%%%%%%%%%%%%%%%%%%
% Programming/Coding Assignment
% LaTeX Template
%
% This template has been downloaded from:
% http://www.latextemplates.com
%
% Original author:
% Ted Pavlic (http://www.tedpavlic.com)
%
% Note:
% The \lipsum[#] commands throughout this template generate dummy text
% to fill the template out. These commands should all be removed when
% writing assignment content.
%
% This template uses a Perl script as an example snippet of code, most other
% languages are also usable. Configure them in the "CODE INCLUSION
% CONFIGURATION" section.
%
%%%%%%%%%%%%%%%%%%%%%%%%%%%%%%%%%%%%%%%%%

%----------------------------------------------------------------------------------------
%	PACKAGES AND OTHER DOCUMENT CONFIGURATIONS
%----------------------------------------------------------------------------------------

\documentclass{article}

\usepackage{fancyhdr} % Required for custom headers
\usepackage{lastpage} % Required to determine the last page for the footer
\usepackage{extramarks} % Required for headers and footers
\usepackage[usenames,dvipsnames]{color} % Required for custom colors
\usepackage{graphicx} % Required to insert images
\usepackage{subcaption}
\usepackage{listings} % Required for insertion of code
\usepackage{courier} % Required for the courier font
\usepackage{lipsum} % Used for inserting dummy 'Lorem ipsum' text into the template
\usepackage{placeins}

% Margins
\topmargin=-0.45in
\evensidemargin=0in
\oddsidemargin=0in
\textwidth=6.5in
\textheight=9.0in
\headsep=0.25in

\linespread{1.1} % Line spacing

% Set up the header and footer
\pagestyle{fancy}
\lhead{\hmwkAuthorName} % Top left header
\chead{\hmwkClass\ (\hmwkClassTime): \hmwkTitle} % Top center head
%\rhead{\firstxmark} % Top right header
\lfoot{\lastxmark} % Bottom left footer
\cfoot{} % Bottom center footer
\rfoot{Page\ \thepage\ of\ \protect\pageref{LastPage}} % Bottom right footer
\renewcommand\headrulewidth{0.4pt} % Size of the header rule
\renewcommand\footrulewidth{0.4pt} % Size of the footer rule

\setlength\parindent{0pt} % Removes all indentation from paragraphs

%----------------------------------------------------------------------------------------
%	CODE INCLUSION CONFIGURATION
%----------------------------------------------------------------------------------------

\definecolor{MyDarkGreen}{rgb}{0.0,0.4,0.0} % This is the color used for comments
\lstloadlanguages{Perl} % Load Perl syntax for listings, for a list of other languages supported see: ftp://ftp.tex.ac.uk/tex-archive/macros/latex/contrib/listings/listings.pdf
\lstset{language=Python, % Use Perl in this example
        frame=single, % Single frame around code
        basicstyle=\small\ttfamily, % Use small true type font
        keywordstyle=[1]\color{Blue}\bf, % Perl functions bold and blue
        keywordstyle=[2]\color{Purple}, % Perl function arguments purple
        keywordstyle=[3]\color{Blue}\underbar, % Custom functions underlined and blue
        identifierstyle=, % Nothing special about identifiers
        breaklines=true, % no text overflow
        commentstyle=\usefont{T1}{pcr}{m}{sl}\color{MyDarkGreen}\small, % Comments small dark green courier font
        stringstyle=\color{Purple}, % Strings are purple
        showstringspaces=false, % Don't put marks in string spaces
        tabsize=5, % 5 spaces per tab
        %
        % Put standard Perl functions not included in the default language here
        morekeywords={rand},
        %
        % Put Perl function parameters here
        morekeywords=[2]{on, off, interp},
        %
        % Put user defined functions here
        morekeywords=[3]{test},
       	%
        morecomment=[l][\color{Blue}]{...}, % Line continuation (...) like blue comment
        numbers=left, % Line numbers on left
        firstnumber=1, % Line numbers start with line 1
        numberstyle=\tiny\color{Blue}, % Line numbers are blue and small
        stepnumber=5 % Line numbers go in steps of 5
}

% Creates a new command to include a perl script, the first parameter is the filename of the script (without .pl), the second parameter is the caption
\newcommand{\perlscript}[2]{
\begin{itemize}
\item[]\lstinputlisting[caption=#2,label=#1]{#1.pl}
\end{itemize}
}

\captionsetup{justification=raggedright,singlelinecheck=false}
%----------------------------------------------------------------------------------------
%	DOCUMENT STRUCTURE COMMANDS
%	Skip this unless you know what you're doing
%----------------------------------------------------------------------------------------

% Header and footer for when a page split occurs within a problem environment
\newcommand{\enterProblemHeader}[1]{
%\nobreak\extramarks{#1}{#1 continued on next page\ldots}\nobreak
%\nobreak\extramarks{#1 (continued)}{#1 continued on next page\ldots}\nobreak
}

% Header and footer for when a page split occurs between problem environments
\newcommand{\exitProblemHeader}[1]{
%\nobreak\extramarks{#1 (continued)}{#1 continued on next page\ldots}\nobreak
%\nobreak\extramarks{#1}{}\nobreak
}

\setcounter{secnumdepth}{1} % Removes default section numbers
\newcounter{homeworkProblemCounter} % Creates a counter to keep track of the number of problems
\setcounter{homeworkProblemCounter}{0}

\newcommand{\homeworkProblemName}{}
\newenvironment{homeworkProblem}[1][Problem \arabic{homeworkProblemCounter}]{ % Makes a new environment called homeworkProblem which takes 1 argument (custom name) but the default is "Problem #"
\stepcounter{homeworkProblemCounter} % Increase counter for number of problems
\renewcommand{\homeworkProblemName}{#1} % Assign \homeworkProblemName the name of the problem
\section{\homeworkProblemName} % Make a section in the document with the custom problem count
\enterProblemHeader{\homeworkProblemName} % Header and footer within the environment
}{
\exitProblemHeader{\homeworkProblemName} % Header and footer after the environment
}

\newcommand{\problemAnswer}[1]{ % Defines the problem answer command with the content as the only argument
\noindent\framebox[\columnwidth][c]{\begin{minipage}{0.98\columnwidth}#1\end{minipage}} % Makes the box around the problem answer and puts the content inside
}

\newcommand{\homeworkSectionName}{}
\newenvironment{homeworkSection}[1]{ % New environment for sections within homework problems, takes 1 argument - the name of the section
\renewcommand{\homeworkSectionName}{#1} % Assign \homeworkSectionName to the name of the section from the environment argument
\subsection{\homeworkSectionName} % Make a subsection with the custom name of the subsection
\enterProblemHeader{\homeworkProblemName\ [\homeworkSectionName]} % Header and footer within the environment
}{
\enterProblemHeader{\homeworkProblemName} % Header and footer after the environment
}

%----------------------------------------------------------------------------------------
%	NAME AND CLASS SECTION
%----------------------------------------------------------------------------------------

\newcommand{\hmwkTitle}{Face Classification with Regression} % Assignment title
\newcommand{\hmwkDueDate}{Friday,\ January\ 29,\ 2018} % Due date
\newcommand{\hmwkClass}{CSC411 Project 1} % Course/class
\newcommand{\hmwkClassTime}{L0101} % Class/lecture time
\newcommand{\hmwkAuthorName}{Hao Hui Tan} % Your name

%----------------------------------------------------------------------------------------
%	TITLE PAGE
%----------------------------------------------------------------------------------------

\title{
\vspace{2in}
\textmd{\textbf{\hmwkClass:\ \hmwkTitle}}\\
\normalsize\vspace{0.1in}\small{Due\ on\ \hmwkDueDate}\\
\vspace{0.1in}
\vspace{3in}
}

\author{\textbf{\hmwkAuthorName}}
%\date{} % Insert date here if you want it to appear below your name

%----------------------------------------------------------------------------------------

\begin{document}

\maketitle
\clearpage
%----------------------------------------------------------------------------------------
%	PROBLEM 1
%----------------------------------------------------------------------------------------

% To have just one problem per page, simply put a \clearpage after each problem

\begin{homeworkProblem}

\noindent \textit{Dataset description}

I have downloaded $1691$ uncropped images of well-known actors and actresses from FaceScrub using the \textbf{get\_data.py} script.  Figures 1, 2 and 3 shows the uncropped images of Peri Gilpin, Steve Carell and Lorrain Bracco respectively. Figures 4, 5 and 6 show the cropped images of these individuals. In the \textbf{get\_data.py} script I have transformed the raw images in the uncropped folder to grayscale, resized them to 32x32 and cropped them to the faces using the bounded box values. \newline

Due to the unfortunate use of \textbf{imsave} function from the \textbf{pylab} library in the \textbf{get\_data.py} script, the cropped
images I got was of the shape (32, 32, 4) instead of (32, 32). As a result,
when these images are opened using imread I got 4096 pixels instead of 1024 (32x32) due to the extra channel the cropped images has. To remedy the situation, whenever I read the cropped images, I always slice the images through [:, :, 0] in order to get rid of the extra channel and read the images as 32x32 for the purpose of this assignment. \newline

As one can see, the cropped out images and the bounding boxes are fairly accurate in cropping out the faces. The bounding boxes values are slightly off in figure 6 because the person in the picture is not facing straight to the camera. The cropped out faces can be aligned with each other for figure 4 and figure 5. Figure 6, as mentioned, is slightly off to the left of the centre than those 2 images.


\begin{figure}[h!]
    \includegraphics[scale=0.5]{gilpin0.jpg}
    \caption{Peri Gilpin - uncropped}
    \label{fig:gilpin}
\end{figure}

\begin{figure}[h!]
    \includegraphics[scale=0.5]{carell1.jpg}
    \caption{Steve Carell - uncropped}
    \label{fig:carell}
\end{figure}

\begin{figure}[h!]
    \includegraphics[scale=0.5]{bracco0.jpg}
    \caption{Lorraine Bracco - uncropped}
    \label{fig:bracco}
\end{figure}

\begin{figure}[h!]
    \includegraphics[scale=1]{cropped/gilpin0.jpg}
    \caption{Peri Gilpin - cropped}
    \label{fig:gilpin_cropped}
\end{figure}

\begin{figure}[h!]
    \includegraphics[scale=1]{cropped/carell1.jpg}
    \caption{Steve Carell - cropped}
    \label{fig:carell_cropped}
\end{figure}

\begin{figure}[h!]
    \includegraphics[scale=1]{cropped/bracco0.jpg}
    \caption{Lorraine Bracco - cropped}
    \label{fig:bracco_cropped}
\end{figure}

\end{homeworkProblem}
\clearpage
%----------------------------------------------------------------------------------------
%	PROBLEM 2
%----------------------------------------------------------------------------------------

\begin{homeworkProblem}
\noindent \textit{Applying an algorithm: initial exploration.}

I have used the shuffle\_images function to separate the images into training set, validation set and test set. I have used dictionaries to store these images where the key is the actor or actress' last name and the value corresponding to the key is the list of his/her images. There are 70 images in the training set, 10 in validation set and 10 in the test set. My algorithm to achieve such a goal is to first put all the cropped image names into an array and then use \textbf{np.random.shuffle()} function to ensure they are out of order. Then I would iterate over all the actors/actresses, and for each actor/actress, I would iterate through all the images available and check if the actor/actress name is in each image's name. If it is, I would add the image to the list of images in the dictionaries with the actor/actress's last name being the dictionary key.

\end{homeworkProblem}
\clearpage
%----------------------------------------------------------------------------------------
%	PROBLEM 3
%----------------------------------------------------------------------------------------

\begin{homeworkProblem}
\noindent \textbf{The cost function that I minimized}

\[J(\theta) = \sum_{i=1}^{m} (h_{\theta}(x^{i}) - y^{i})^{2}\]

where $\theta$ is the vector of weights we seek to minimize
      $x^{i}$ is the set of pixel features of the $i^{th}$ image sample
      $m$ is the number of image samples
      $h_{\theta}(x^{i})$ is the hypothesis function with respect to $x^{i}$
      $y^{i}$ is the label of the $i^{th}$ image sample where $y^{i} = 1$ represents the image is Alec Baldwin and $y^{i} = 0$ represents the image is Steve Carell
\newline

The following values can be generated and seen in the \textbf{part\_3\_data\_file.txt}:
\newline
The cost function value for the training set is 3.73786191455 \newline
The cost function value for the validation set is 2.41870583942 \newline
The performance of the classifier (percent accuracy) on the training set is 1.0 \newline
The performance of the classifier (percent accuracy) on the validation set is 0.85
\newline

\begin{lstlisting}
def binary_classify(performer_1_full, performer_2_full,  training_samples_num=70, alpha=0.0000010, init_theta_coefficient=0):
    """Inputs are 2 actors' last names to classify
       Returns the trained thetas array for the hypothesis and the cost function values
       for the training and validation sets
    """
    performer_1 = performer_1_full.split(" ")[1].lower()
    performer_2 = performer_2_full.split(" ")[1].lower()
    if performer_1_full in actors:
        path_1 = "cropped/actors/"
    else:
        path_1 = "cropped/actresses/"

    if performer_2_full in actors:
        path_2 = "cropped/actors/"
    else:
        path_2 = "cropped/actresses/"
    # Training images for alec baldwin and steve Carel
    performer_1_training_set = training_dictionary[performer_1]
    performer_2_training_set = training_dictionary[performer_2]
    performer_1_validation_set = validation_dictionary[performer_1]
    performer_2_validation_set = validation_dictionary[performer_2]
    training_set = []
    validation_set = []
    performer_1_num_images = 0
    performer_2_num_images = 0
    performer_1_num_images_validation = 0
    performer_2_num_images_validation = 0

    for image_name in performer_1_training_set:
        if performer_1_num_images > training_samples_num:
            break
        path = path_1 + image_name
        image_file = imread(path)[:, :, 0]
        # Get the flatten image of inputs
        flatten_image = image_file.flatten()
        flatten_image_processed = flatten_image / 255.0  # so that each input is between 0 and 1
        training_set.append(flatten_image_processed) # training set 2D array
        performer_1_num_images = performer_1_num_images + 1

    for image_name in performer_2_training_set:
        if performer_1_num_images > training_samples_num:
            break
        path = path_2 + image_name
        image_file = imread(path)[:, :, 0]
        # Get the flatten image of inputs
        flatten_image = image_file.flatten()
        flatten_image_processed = flatten_image / 255.0  # so that each input is between 0 and 1
        training_set.append(flatten_image_processed) # training set 2D array
        performer_2_num_images = performer_2_num_images + 1

    for image_name in performer_1_validation_set:
        if performer_1_num_images_validation > training_samples_num:
            break
        path = path_1 + image_name
        image_file = imread(path)[:, :, 0]
        # Get the flatten image of inputs
        flatten_image = image_file.flatten()
        flatten_image_processed = flatten_image / 255.0  # so that each input is between 0 and 1
        validation_set.append(flatten_image_processed) # training set 2D array
        performer_1_num_images_validation = performer_1_num_images_validation + 1

    for image_name in performer_2_validation_set:
        if performer_2_num_images_validation > training_samples_num:
            break
        path = path_1 + image_name
        image_file = imread(path)[:, :, 0]
        # Get the flatten image of inputs
        flatten_image = image_file.flatten()
        flatten_image_processed = flatten_image / 255.0  # so that each input is between 0 and 1
        validation_set.append(flatten_image_processed) # training set 2D array
        performer_2_num_images_validation = performer_2_num_images_validation + 1

    x_matrix = np.vstack(training_set) # input x matrix for gradient descent. 70 rows = 70 images. 4096 columns = pixels
    x_matrix_validation = np.vstack(validation_set)
    # print "sum of x_matrix is:", sum(x_matrix)
    # print "init_theta_coefficient", init_theta_coefficient
    initial_theta = init_theta_coefficient * np.ones(len(x_matrix[0])) # initial thetas for graident descent
    y_vector = np.ones(performer_1_num_images) # 1 for alec baldwin, 0 for Steve Carell
    y_vector = np.append(y_vector, np.zeros(performer_2_num_images))

    y_vector_validation = np.append(np.ones(performer_1_num_images_validation), np.zeros(performer_2_num_images_validation))
    result_theta = grad_descent(f, df, x_matrix.T, y_vector, initial_theta, alpha)
    result_cost_training = f(x_matrix.T, y_vector, result_theta)
    result_cost_validation = f(x_matrix_validation.T, y_vector_validation, result_theta)
    return result_theta, result_cost_training, result_cost_validation
\end{lstlisting}

\textbf{Description of the process to get the code to work}
\newline
I am using the python functions $f$, $df$ and $grad_descent$ from the tutorial code. As I am building the classifier, there were dimension misalignment issues. I had to use the x\_matrix transpose to solve the issue. At first, my alpha was 0.1. I run into the RuntimeWarning: overflow encountered in multiply error. It seems to suggest that the alpha is too large and as a result gradient descent overshoots the minima multiple times and as a result it could not converge. I began to experiment with smaller alphas and eventually chose the alpha value of 0.0000010 as the default. I chose it because it is one of the smallest values for the code to not run forever and it gives a decent set of $\theta$ values for the prediction.

\end{homeworkProblem}
\clearpage

%----------------------------------------------------------------------------------------
%	PROBLEM 4
%----------------------------------------------------------------------------------------

\begin{homeworkProblem}
\noindent \textbf{Part a)}

\begin{figure}[h!]
    \includegraphics[scale=1]{part_4a_full_training.jpg}
    \caption{Thetas obtained from training using the full training set}
    \label{fig:full_training}
\end{figure}

\begin{figure}[h!]
    \includegraphics[scale=1]{part_4a_2_samples_training.jpg}
    \caption{Thetas obtained from training using 2 samples from the training set}
    \label{fig:two_samples}
\end{figure}

\begin{figure}[h!]
    \includegraphics[scale=1]{part_4b_foggy_face.jpg}
    \caption{Thetas obtained from training using full training set}
    \label{fig:foggy_face}
\end{figure}

\begin{figure}[h!]
    \includegraphics[scale=1]{part_4b_new_start_theta.jpg}
    \caption{Thetas obtained from training using full training set but with different start thetas}
    \label{fig:new_start_theta}
\end{figure}

In figure 10, I used a new set of initial $\theta$ values of 0.5 instead of zeros. In figure 10, I just followed the normal procedure to produce the image.


\end{homeworkProblem}
\clearpage

%----------------------------------------------------------------------------------------

%	PROBLEM 5
%----------------------------------------------------------------------------------------
\begin{homeworkProblem}

\noindent \textbf{Performance of the gender classifiers on the training and validation sets vs the size of the training set.}

Note: For this training problem, I use $y = 1$ to denote the performer in the image is male and $y = 0$ to denote the image is female.

\begin{figure}[h!]
    \includegraphics[scale=1]{part_5_validation_training_plot.png}
    \caption{performance of the classifiers on the training and validation sets vs the size of the training set.}
    \label{fig:performance_gender}
\end{figure}

The following is the performance of the 6 actors who are not included in the original \textbf{act} set. The results are generated and seen in \textbf{part\_5\_different\_performer\_accuracies.txt} file.
\newline

Gerard Butler Accuracy is: 0.992857142857 \newline
Fran Drescher Accuracy is: 0.882352941176 \newline
Michael Vartan Accuracy is: 0.68085106383 \newline
America Ferrera Accuracy is: 0.955801104972 \newline
Daniel Radcliffe Accuracy is: 0.795918367347 \newline
Kristin Chenoweth Accuracy is: 0.766081871345 \newline

As one can see, there are clear signs of overfitting as the accuracies go down when one test the hypothesis on different actors and actresses that were not in the original actors and actresses sets for training.


\end{homeworkProblem}
%----------------------------------------------------------------------------------------

\end{document}
